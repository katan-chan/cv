\section{Trích xuất chuyển động từ ảnh}

Sau khi các landmarks đã được xác định bởi MediaPipe (như trình bày trong Chương 3),
bước kế tiếp là trích xuất thông tin về sự thay đổi vị trí của chúng theo thời gian.  
Mục tiêu của giai đoạn này là xây dựng một biểu diễn định lượng cho chuyển động của bàn tay,
thông qua các đại lượng như \textbf{vector dịch chuyển}, \textbf{vận tốc}, và \textbf{hướng chuyển động}.

Quá trình này được thực hiện bằng thuật toán \textbf{Lucas–Kanade Optical Flow}~\cite{lucas1981iterative}
trong không gian pixel.
Trước khi áp dụng, landmarks chuẩn hóa từ MediaPipe được chuyển về toạ độ pixel:
\[
(u_i, v_i) = (x_i \cdot W,\, y_i \cdot H),
\]
với $W = 1280$, $H = 720$ là kích thước khung hình.
\subsection{Nguyên lý và mô hình toán học}

Giả sử $I(x,y,t)$ là cường độ sáng của điểm ảnh tại thời điểm $t$.  
Khi bàn tay di chuyển, cùng một điểm vật lý trong không gian sẽ xuất hiện tại vị trí $(x+u, y+v)$
ở khung hình kế tiếp $t+\Delta t$, với $(u,v)$ là vector dịch chuyển cục bộ.  
Giả định cường độ sáng không đổi dẫn tới ràng buộc:
\[
I(x, y, t) = I(x + u, y + v, t + \Delta t).
\]
Khai triển Taylor và bỏ qua các bậc cao hơn, ta thu được phương trình quang học cơ bản:
\[
\frac{\partial I}{\partial x}u +
\frac{\partial I}{\partial y}v +
\frac{\partial I}{\partial t} = 0.
\]
Đây là nền tảng của phương pháp Lucas–Kanade,
trong đó $(u,v)$ được coi là không đổi trong một cửa sổ nhỏ quanh mỗi điểm mốc,
và được xác định bằng cách cực tiểu hoá sai số bình phương
giữa hai khung hình liên tiếp.

\subsection{Phương pháp Lucas–Kanade cục bộ}

Tại mỗi landmark $(x_i, y_i)$,
ta xét tập hợp các điểm $(x_j, y_j)$ trong cửa sổ kích thước $k \times k$ xung quanh nó.
Hệ phương trình tuyến tính được thiết lập:
\[
A_i
\begin{bmatrix}
u_i \\ v_i
\end{bmatrix}
 = -b_i,
\qquad
A_i =
\begin{bmatrix}
\frac{\partial I}{\partial x}(x_1, y_1) & \frac{\partial I}{\partial y}(x_1, y_1) \\
\vdots & \vdots \\
\frac{\partial I}{\partial x}(x_n, y_n) & \frac{\partial I}{\partial y}(x_n, y_n)
\end{bmatrix},
\quad
b_i =
\begin{bmatrix}
\frac{\partial I}{\partial t}(x_1, y_1) \\ \vdots \\ \frac{\partial I}{\partial t}(x_n, y_n)
\end{bmatrix}.
\]
Vector dịch chuyển $(u_i, v_i)$ được tính bằng nghiệm tối thiểu bình phương:
\[
\begin{bmatrix}
u_i \\ v_i
\end{bmatrix}
 = - (A_i^\top A_i)^{-1} A_i^\top b_i.
\]
Các giá trị $(u_i, v_i)$ biểu diễn chuyển động biểu kiến của landmark $i$
trên mặt phẳng ảnh giữa hai khung hình liên tiếp.

\subsection{Tính vận tốc và hướng chuyển động}

Từ các giá trị dịch chuyển $(u_i, v_i)$,
ta có thể suy ra độ lớn và hướng của chuyển động:
\[
\text{speed}_i = \sqrt{u_i^2 + v_i^2}, \qquad
\text{angle}_i = \operatorname{atan2}(v_i, u_i).
\]
Để giảm nhiễu, các vector dịch chuyển được làm trơn bằng trung bình trượt trên $N$ khung hình gần nhất:
\[
\bar{u}_i = \frac{1}{N}\sum_{k=1}^{N} u_{i,k}, \qquad
\bar{v}_i = \frac{1}{N}\sum_{k=1}^{N} v_{i,k},
\]
với $N \le 5$.  
Từ đó, vận tốc và hướng trung bình được xác định:
\[
\overline{\text{speed}}_i = \sqrt{\bar{u}_i^2 + \bar{v}_i^2}, \qquad
\overline{\text{angle}}_i = \operatorname{atan2}(\bar{v}_i, \bar{u}_i).
\]
Nếu tại một khung hình optical flow không thu được kết quả hợp lệ,
giá trị của landmark được giữ nguyên hoặc thay bằng trung bình gần nhất
để đảm bảo tính liên tục của quỹ đạo.


\section{Trích xuất phép quay giữa các khung hình}

Sau khi có landmarks 3D (đã dịch về gốc wrist như mô tả ở Chương 3),
bước tiếp theo là ước lượng sự thay đổi định hướng (orientation) của bàn tay
giữa hai khung hình liên tiếp.  

Trong nghiên cứu này, phép quay toàn cục (root rotation) được trích xuất
bằng phương pháp \textbf{Kabsch}~\cite{kabsch1976solution},
một kỹ thuật tuyến tính ổn định cho bài toán căn chỉnh cứng (rigid alignment).
Để giảm nhiễu và tập trung vào cấu trúc lòng bàn tay,
ta chỉ sử dụng \textbf{ba điểm đại diện} thay vì toàn bộ 21 landmarks.

\subsection{Mô hình bài toán}

Gọi $\mathcal{P} = \{p_W, p_I, p_P\}$ là tập ba điểm đại diện
(wrist, index MCP, pinky MCP tương ứng với ID $\{0, 5, 17\}$)
ở frame đầu tiên (bind pose), và
$\mathcal{Q} = \{q_W, q_I, q_P\}$ là ba điểm tương ứng
ở frame hiện tại,
với các điểm $p_i, q_i \in \mathbb{R}^3$ đã được dịch về gốc wrist.

Mục tiêu là tìm ma trận quay $R \in SO(3)$ 
sao cho sai số bình phương giữa hai tập điểm được tối thiểu hoá:
\[
E(R) = \sum_{i \in \{W,I,P\}} \| q_i - R p_i \|^2.
\]
Do đã dịch về gốc wrist, thành phần tịnh tiến $t = 0$.

\subsection{Giải thuật Kabsch}

Bài toán trên được giải bằng cách trước hết đưa hai tập điểm về cùng trọng tâm (centroid):
\[
\tilde{p}_i = p_i - \bar{p}, \qquad 
\tilde{q}_i = q_i - \bar{q},
\]
trong đó $\bar{p} = \frac{1}{3}(p_W + p_I + p_P)$ và $\bar{q} = \frac{1}{3}(q_W + q_I + q_P)$
là trọng tâm của tam giác tạo bởi ba điểm.

Ma trận hiệp phương sai (cross-covariance) được xác định bởi:
\[
H = \tilde{P}^\top \tilde{Q}
= \begin{bmatrix}
\tilde{p}_W^\top \\ \tilde{p}_I^\top \\ \tilde{p}_P^\top
\end{bmatrix}
\begin{bmatrix}
\tilde{q}_W & \tilde{q}_I & \tilde{q}_P
\end{bmatrix}
\in \mathbb{R}^{3 \times 3}.
\]

Thực hiện phân rã giá trị rieng suy biến (SVD):
\[
H = U S V^\top.
\]
Khi đó, ma trận quay tối ưu được tính bằng~\cite{kabsch1976solution}:
\[
R = V U^\top.
\]
Nếu $\det(R) < 0$, 
cột cuối của $V$ được đổi dấu để đảm bảo $R \in SO(3)$.

\subsection{Ý nghĩa hình học}

Ma trận $R$ biểu diễn sự quay toàn cục (root rotation) của bàn tay
giữa frame hiện tại so với bind pose ban đầu.
Đây chính là thông tin định hướng (orientation) của toàn bộ bàn tay trong không gian 3D.

Trong pipeline của hệ thống:
\begin{itemize}
    \item Ma trận $R$ được chuyển thành quaternion $\mathbf{q}_{\text{root}}$ để truyền đến client 3D.
    \item Quaternion này được làm mượt bằng SLERP (Spherical Linear Interpolation)~\cite{shoemake1985animating}
          để giảm nhiễu cao tần giữa các frame liên tiếp.
    \item Kết quả được sử dụng làm root rotation cho skeleton trong Chương 5.
\end{itemize}
