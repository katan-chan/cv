\section{Không gian và hệ tọa độ}

Xét không gian Euclid ba chiều $\mathbb{R}^3$ với hệ tọa độ Descartes có gốc tại $O(0,0,0)$.  
Mỗi điểm trong không gian được biểu diễn bởi một vectơ vị trí 
\[
p = (x,\, y,\, z)^\top.
\]

Mọi chuyển động cứng trong không gian có thể được mô tả bằng hai phép biến đổi cơ bản:

\begin{itemize}
    \item \textbf{Phép tịnh tiến}: dịch toàn bộ hệ theo một vectơ $t = (t_x, t_y, t_z)^\top$, làm thay đổi vị trí nhưng không làm thay đổi hướng hay hình dạng.
    \item \textbf{Phép quay}: quay toàn bộ hệ quanh một trục cố định đi qua gốc tọa độ, được mô tả bởi ma trận quay $R \in \mathbb{R}^{3\times3}$.
\end{itemize}

Giả sử tại thời điểm $t$, tọa độ của một điểm là $p = (x, y, z)^\top$.  
Khi đó, tác động tổng hợp của phép quay $R$ và phép tịnh tiến $t$ được biểu diễn bằng hàm biến đổi:
\[
f(p) = R p + t.
\]


\section{Phép tịnh tiến}

Xét không gian Euclid ba chiều $\mathbb{R}^3$ với hệ tọa độ Descartes $(Oxyz)$.  
\textbf{Phép tịnh tiến} là phép biến đổi dịch chuyển toàn bộ không gian 
theo cùng một hướng và cùng một độ dài, không làm thay đổi hình dạng, kích thước hay hướng của vật thể.

Phép tịnh tiến được xác định bởi một vectơ dịch chuyển 
\[
t = (t_x, t_y, t_z)^\top,
\]
trong đó $t_x$, $t_y$, $t_z$ lần lượt là độ dời theo các trục $x$, $y$, $z$.

Nếu một điểm ban đầu có tọa độ $p = (x, y, z)^\top$, 
thì sau khi thực hiện phép tịnh tiến, điểm mới $p'$ được xác định bởi:
\[
p' = p + t =
\begin{bmatrix}
x + t_x\\[3pt]
y + t_y\\[3pt]
z + t_z
\end{bmatrix}.
\]

\section{Phép xoay}

Xét không gian Euclid ba chiều $\mathbb{R}^3$ với hệ tọa độ Descartes $(Oxyz)$, 
trong đó $O(0,0,0)$ là gốc tọa độ.

\textbf{Phép xoay (rotation)} là phép biến đổi bảo toàn gốc tọa độ và độ dài của mọi vectơ, 
đồng thời thay đổi hướng của chúng.  
Một phép xoay được xác định bởi một trục quay và một góc quay quanh trục đó.  
Tập hợp tất cả các phép xoay trong $\mathbb{R}^3$ tạo thành nhóm đặc biệt trực giao $SO(3)$.

Trong thực hành, ta thường mô tả một phép xoay tổng quát bằng ba phép xoay liên tiếp quanh 
các trục tọa độ chính $x$, $y$ và $z$.  
Ba góc quay tương ứng $\theta_x$, $\theta_y$, $\theta_z$ 
được gọi là \textbf{các góc Euler}, lần lượt biểu diễn:

\begin{itemize}
    \item $\theta_x$: góc quay quanh trục $x$, điều khiển sự \emph{nghiêng} lên xuống ;
    \item $\theta_y$: góc quay quanh trục $y$, điều khiển sự \emph{quay ngang} sang hai bên ;
    \item $\theta_z$: góc quay quanh trục $z$, điều khiển sự \emph{xoay tròn} quanh trục dọc .
\end{itemize}

Ba phép quay cơ bản này được biểu diễn bằng các ma trận:

\subsection*{Phép quay quanh trục $x$}
\[
R_x(\theta_x) =
\begin{bmatrix}
1 & 0 & 0\\
0 & \cos\theta_x & -\sin\theta_x\\
0 & \sin\theta_x & \cos\theta_x
\end{bmatrix}.
\]

\subsection*{Phép quay quanh trục $y$}
\[
R_y(\theta_y) =
\begin{bmatrix}
\cos\theta_y & 0 & \sin\theta_y\\
0 & 1 & 0\\
-\sin\theta_y & 0 & \cos\theta_y
\end{bmatrix}.
\]

\subsection*{Phép quay quanh trục $z$}
\[
R_z(\theta_z) =
\begin{bmatrix}
\cos\theta_z & -\sin\theta_z & 0\\
\sin\theta_z & \cos\theta_z & 0\\
0 & 0 & 1
\end{bmatrix}.
\]

\subsection*{Phép quay tổng quát}
Một phép quay tổng quát trong không gian 3 chiều có thể được biểu diễn 
dưới dạng hợp của ba phép quay cơ bản:
\[
R = R_z(\theta_z)\, R_y(\theta_y)\, R_x(\theta_x),
\]
trong đó thứ tự nhân ma trận xác định hệ quy chiếu (fixed-frame hoặc moving-frame).  

\section{Phân rã giá trị riêng suy biến (SVD)}

Với mọi ma trận $H \in \mathbb{R}^{m \times n}$, 
tồn tại các ma trận trực giao $U \in \mathbb{R}^{m \times m}$ và $V \in \mathbb{R}^{n \times n}$, 
cùng ma trận chéo $S \in \mathbb{R}^{m \times n}$ có các phần tử không âm sao cho:
\[
H = U S V^\top.
\]
Các phần tử trên đường chéo của $S$ gọi là \textbf{các giá trị suy biến} (singular values) của $H$.

\section{Biểu diễn bằng quaternion}

Quaternion là phần mở rộng của số phức vào không gian ba chiều, 
có dạng tổng quát:
\[
q = w + x\,\mathbf{i} + y\,\mathbf{j} + z\,\mathbf{k},
\]
trong đó $w, x, y, z \in \mathbb{R}$ và các đơn vị $\mathbf{i}, \mathbf{j}, \mathbf{k}$ thỏa mãn:
\[
\mathbf{i}^2 = \mathbf{j}^2 = \mathbf{k}^2 = \mathbf{i}\mathbf{j}\mathbf{k} = -1.
\]

Một \textbf{quaternion đơn vị} $q = (x, y, z, w)$ với $\|q\| = 1$ 
biểu diễn một phép quay trong không gian 3 chiều.  
Cụ thể, phép quay quanh trục đơn vị $\hat{u} = (u_x, u_y, u_z)$ 
với góc quay $\theta$ được viết dưới dạng:
\[
q = \cos\frac{\theta}{2} + \hat{u}\,\sin\frac{\theta}{2}
= \cos\frac{\theta}{2} + (u_x\mathbf{i} + u_y\mathbf{j} + u_z\mathbf{k})\,\sin\frac{\theta}{2}.
\]

\begin{figure}
    \centering
    \includegraphics[width=0.5\textwidth]{figures/axisAngle1.png}
    \caption{Ví dụ mô phỏng phép quay bằng quaternion.}
\end{figure}