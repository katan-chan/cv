\section{Không gian và cấu trúc khung xương của mô hình bàn tay}

Mô hình bàn tay được biểu diễn dưới dạng hệ xương phân cấp (skeletal hierarchy),
trong đó mỗi khớp (bone) được mô tả bởi ba thành phần:
tên khớp, khớp cha (\texttt{parent}), 
vị trí tương đối so với khớp cha (\texttt{position}) 
và quaternion quay cục bộ (\texttt{quaternion}).
Tập dữ liệu \texttt{skeleton\_metadata.json} chứa tổng cộng 26 nút:
một nút gốc \texttt{\_rootJoint}, 
một nút trung tâm \texttt{Bone\_00},
và 24 khớp còn lại thuộc năm ngón tay.

\begin{figure}
    \centering
    \includegraphics[width=0.6\textwidth]{figs/hand_skeleton.png}
    \caption{Cấu trúc khung xương bàn tay trong mô hình 3D}
    \label{fig:hand_skeleton}
\end{figure}

\subsection{Hệ toạ độ cơ sở}

Nút gốc \texttt{\_rootJoint} có
\[
\texttt{position} = (0, 0, 0), \qquad \texttt{quaternion} = (0, 0, 0, 1),
\]
được dùng làm gốc của toàn bộ hệ toạ độ.
Khớp \texttt{Bone\_00} là con trực tiếp của \texttt{\_rootJoint}
với vị trí \((0, 0, 0)\) và một quaternion quay cố định,
đóng vai trò là khớp cổ tay trung tâm.

Mọi khớp còn lại được biểu diễn trong hệ toạ độ local của khớp cha:
vector \texttt{position} cho biết hướng và độ dài của đoạn xương tiếp theo,
còn \texttt{quaternion} mô tả hướng xoay của đoạn đó trong tư thế bind.
Độ dài các đoạn xương (chuẩn hoá) được tính từ chuẩn Euclid của vector vị trí
và nằm trong khoảng:
\[
0.60 \lesssim \|\texttt{position}\| \lesssim 3.23.
\]
Tổng độ dài từ cổ tay tới đầu ngón giữa (chuỗi \texttt{Bone008\_08} đến \texttt{Bone011\_end\_022})
xấp xỉ $7.25$ đơn vị ảo.

\subsection{Cấu trúc phân cấp của khung xương}

Các khớp được tổ chức thành năm chuỗi tương ứng với năm ngón tay,
đều xuất phát từ \texttt{Bone\_00}.
Tên khớp mang phần hậu tố dạng \texttt{\_NN},
trong đó \texttt{NN} tương ứng với chỉ số landmark trong hệ 3D bên ngoài.

Bảng~\ref{tab:hand_skeleton} tóm tắt cấu trúc cho từng ngón tay,
bao gồm số đoạn liên tiếp và khoảng độ dài tương đối của các đoạn
(theo dữ liệu trong \texttt{skeleton\_metadata.json}).

\begin{table}[h!]
\centering
\caption{Khung xương bàn tay (tóm tắt từ \texttt{skeleton\_metadata.json})}
\label{tab:hand_skeleton}
\renewcommand{\arraystretch}{1.05}
\begin{tabular}{|l|c|c|}
\hline
\textbf{Ngón tay} & \textbf{Số đoạn (bao gồm end)} & \textbf{Khoảng chiều dài} \\ \hline
Ngón cái (Thumb)  & 4 & $0.91\,\text{–}\,2.11$ \\ \hline
Ngón trỏ (Index)  & 5 & $0.77\,\text{–}\,3.11$ \\ \hline
Ngón giữa (Middle) & 5 & $0.93\,\text{–}\,3.11$ \\ \hline
Ngón áp út (Ring) & 5 & $0.87\,\text{–}\,3.21$ \\ \hline
Ngón út (Pinky)   & 5 & $0.60\,\text{–}\,3.22$ \\ \hline
\end{tabular}
\end{table}


\subsection{Định hướng khớp bằng quaternion}

Mỗi khớp mang một quaternion đơn vị $(x, y, z, w)$
mô tả phép quay local so với khớp cha trong tư thế bind.
Ví dụ:
\begin{equation}
\begin{aligned}
q_{\texttt{Bone001\_01}} &= (6.20\times 10^{-8},\; 2.91\times 10^{-8},\; -0.4092,\; 0.9124), \\
q_{\texttt{Bone002\_02}} &= (-8.5\times 10^{-10},\; -5.2\times 10^{-8},\; -0.0176,\; 0.9998).
\end{aligned}
\end{equation}
Từ các quaternion này, có thể khôi phục ma trận quay $R \in SO(3)$ của từng khớp
và tính được hướng xương (bone direction) trong không gian mô hình.

\subsection{Ý nghĩa trong không gian mô hình}

Không gian của mô hình bàn tay hoàn toàn là không gian chuẩn hoá:
tất cả các toạ độ và độ dài là tương đối,
chỉ giữ tỉ lệ giữa các đốt tay và cấu trúc phân cấp giữa các khớp.
Cấu trúc này cung cấp:
\begin{itemize}
    \item khung tham chiếu hình học để ánh xạ landmarks thu được từ ảnh
          vào hệ xương 3D;
    \item thông tin động học (cha–con) để tính động học thuận;
    \item hướng khớp ban đầu để nội suy quay và thực hiện retargeting chuyển động.
\end{itemize}
Nhờ đó, mô hình có thể nhận chuyển động ước lượng từ ảnh
và biểu diễn lại trong không gian 3D một cách nhất quán.

\section{Retargeting từ landmarks 3D sang mô hình xương}

Các chương trước đã trình bày cách thu thập và xử lý dữ liệu từ camera:
\begin{itemize}
    \item Chương 3: MediaPipe phát hiện 21 landmarks và trả về tọa độ chuẩn hóa.
    \item Chương 4: Optical flow trích xuất chuyển động; Kabsch tính root rotation từ 3 điểm đại diện.
\end{itemize}

Bước cuối cùng là ánh xạ toàn bộ dữ liệu này lên hệ xương phân cấp của mô hình 3D.
Quá trình này gọi là \textbf{retargeting}, bao gồm:
\begin{enumerate}
    \item Chuyển đổi không gian tọa độ (camera $\to$ scene).
    \item Ánh xạ landmarks sang bones thông qua bảng mapping.
    \item Tính quaternion local cho từng khớp dựa trên hướng của landmarks.
    \item Áp dụng forward kinematics để dựng lại skeleton hoàn chỉnh.
\end{enumerate}

\subsection{Cấu trúc phân cấp và bảng ánh xạ}

Skeleton của mô hình bao gồm hai nút gốc đặc biệt:
\texttt{\_rootJoint} (gốc tuyệt đối) và \texttt{Bone\_00} (root bone chính).
Nút \texttt{Bone\_00} có quaternion khởi tạo không trivial:
\[
q_{\texttt{Bone\_00}} = (0.707, 0, 0, 0.707),
\]
thể hiện phép quay $90^\circ$ quanh trục $x$ để căn chỉnh hệ toạ độ mô hình với hệ toạ độ làm việc.

Ánh xạ từ landmarks sang bones không phải là song ánh:
chỉ một tập con các landmarks được chọn để điều khiển các khớp quan trọng.
Bảng ánh xạ $\mathcal{M}$ xác định quan hệ:
\[
\mathcal{M}: \{0, 1, 2, 4, 5, \ldots, 20\} \to \{\texttt{Bone\_00}, \texttt{Bone002\_02}, \ldots\},
\]
ví dụ: $\mathcal{M}(0) = \texttt{Bone\_00}$, 
$\mathcal{M}(1) = \texttt{Bone002\_02}$,
$\mathcal{M}(5) = \texttt{Bone013\_013}$.

Lưu ý rằng một số ID landmarks (như 3, 13, 14) không xuất hiện trong $\mathcal{M}$,
và các bone trung gian (như \texttt{Bone001\_01}, \texttt{Bone004\_04})
được điều khiển gián tiếp qua chuỗi động học.

\subsection{Chuỗi ngón tay và bind pose}

Mỗi ngón tay được biểu diễn bằng một chuỗi các cặp $(i, j)$,
với $i, j$ là ID của landmarks cha–con:
\begin{align*}
\text{Thumb} &: \{(1,2), (2,3), (3,4)\}, \\
\text{Index} &: \{(5,6), (6,7), (7,8)\}, \\
\text{Middle} &: \{(9,10), (10,11), (11,12)\}, \\
\text{Ring} &: \{(13,14), (14,15), (15,16)\}, \\
\text{Pinky} &: \{(17,18), (18,19), (19,20)\}.
\end{align*}

Tại frame đầu tiên khi phát hiện bàn tay, 
bộ ba điểm $\{L_0, L_5, L_{17}\}$ (wrist, index MCP, pinky MCP)
được ghi lại làm bind pose cho thuật toán Kabsch.
Các landmarks còn lại không tham gia vào tính toán root rotation,
mà chỉ dùng để xác định hướng của từng đốt ngón.

\subsection{Chuyển đổi không gian tọa độ}

Landmarks từ MediaPipe nằm trong hệ camera với quy ước:
trục $+x$ sang phải, $+y$ xuống, $+z$ ra khỏi camera.
Để chuyển sang hệ scene (thường dùng $+Y$ lên trong Three.js/Unity), áp dụng ma trận:
\[
R_{\text{scene}} = \begin{bmatrix}
1 & 0 & 0 \\
0 & -1 & 0 \\
0 & 0 & 1
\end{bmatrix}.
\]

Đối với tay trái, cần thêm phép gương theo trục $x$:
\[
R_{\text{mirror}} = \begin{bmatrix}
-1 & 0 & 0 \\
0 & 1 & 0 \\
0 & 0 & 1
\end{bmatrix}.
\]

Phép biến đổi tổng quát từ không gian landmark sang không gian model:
\[
\mathbf{p}_{\text{model}} = R_{\text{scene}} \cdot R_{\text{mirror}} \cdot \mathbf{p}_{\text{camera}},
\]
với $R_{\text{mirror}} = I$ nếu là tay phải.

\subsection{Tính quaternion local cho từng bone}

Với mỗi cặp landmarks liên tiếp $(i, j)$ trong chuỗi ngón tay,
vector hướng được tính bằng:
\[
\mathbf{d}_{ij} = \mathbf{p}_j - \mathbf{p}_i,
\]
sau đó chuyển đổi sang không gian model:
\[
\mathbf{d}'_{ij} = R_{\text{scene}} \cdot R_{\text{mirror}} \cdot \mathbf{d}_{ij}.
\]

Quaternion local của bone tương ứng được xác định bằng phép quay ngắn nhất
từ hướng chuẩn \texttt{bone\_dir} $= (0, 1, 0)^\top$ (hướng dương trục $y$ trong local space)
sang hướng quan sát $\hat{\mathbf{d}}'_{ij}$ (đã chuẩn hóa):
\[
\begin{aligned}
\mathbf{a} &= \texttt{bone\_dir} \times \hat{\mathbf{d}}'_{ij}, \\
\theta &= \arccos(\texttt{bone\_dir} \cdot \hat{\mathbf{d}}'_{ij}), \\
\mathbf{q} &= \left[\mathbf{a} \cdot \sin\left(\tfrac{\theta}{2}\right), \cos\left(\tfrac{\theta}{2}\right)\right].
\end{aligned}
\]

Trường hợp đặc biệt: nếu tay trái, ma trận quay của root bone cũng cần được mirror:
\[
R_{\text{root}}^{\text{left}} = R_{\text{mirror}} \cdot R_{\text{root}}^{\text{right}} \cdot R_{\text{mirror}}.
\]

\subsection{Forward kinematics}

Vị trí global của từng bone được tính tuần tự theo thứ tự phân cấp.
Gọi $\ell_b$ là độ dài rest (trong bind pose) của bone $b$,
vị trí của bone con được xác định:
\[
\mathbf{p}_{\text{child}}^{\text{global}} = \mathbf{p}_{\text{parent}}^{\text{global}} 
+ R_{\text{parent}}^{\text{global}} \cdot (\hat{\mathbf{d}}'_{\text{parent}} \cdot \ell_b),
\]
trong đó $R_{\text{parent}}^{\text{global}}$ là tích lũy các quaternion từ root xuống parent,
và $\hat{\mathbf{d}}'_{\text{parent}}$ là hướng đã chuẩn hóa của bone parent.

Kết quả retargeting bao gồm hai thành phần:
\begin{itemize}
    \item $\mathcal{P}_{\text{global}} = \{\texttt{bone\_name} \mapsto \mathbf{p}^{\text{global}}\}$:
          toạ độ 3D của từng bone trong scene space,
    \item $\mathcal{Q}_{\text{local}} = \{\texttt{bone\_name} \mapsto \mathbf{q}\}$:
          quaternion local cho từng bone.
\end{itemize}

Dữ liệu này được tuần tự hóa thành JSON và truyền qua WebSocket đến client 3D (Three.js)
với tần suất 30 khung hình/giây, cho phép hiển thị chuyển động thời gian thực.
